\documentclass[main]{subfiles}

\begin{document}

% Proportions

\subsection{One Sample Confidence Interval with Proportions}
\textbf{State}
\begin{itemize}
    \item I will construct a one-sample z interval for $p$ where $p = ...$
\end{itemize}
\textbf{Plan}
\begin{itemize}
    \item Random samples
    \item 10\%: $n \leq 0.1N$
    \item Large counts: $n\hat{p} \geq 10$ and $n(1-\hat{p}) \geq 10$
\end{itemize}
\textbf{Do} \textit{(1-PropZInt)}

\[\hat{p} \pm z* \sqrt{\frac{\hat{p}(1-\hat{p})}{n}}\]

\noindent\textbf{Conclude}
I am C\% confident that the interval from ... captures the true population proportion of ...
\\~\\
\noindent\textbf{Using interval:} Since ... was/was not captured in the (C\%) confidence interval ... (for ...), it is/is not plausible that ... Therefore, the confidence interval provides/does not provide convincing evidence that ...
\\~\\
\noindent\textbf{Sample size needed:} $n \geq (\frac{z*}{{ME}})^2 \times \hat{p}(1-\hat{p})$
\\~\\
\noindent\hyperlink{toc}{Back!}
\newline\hrule

% Means

\subsection{One Sample Confidence Interval with Means}
\textbf{State}
\begin{itemize}
    \item I will construct a one-sample t interval for $\mu$ where $\mu = ...$
\end{itemize}
\textbf{Plan}
\begin{itemize}
    \item Random samples
    \item 10\%: $n \leq 0.1N$
    \item Normality/CLT: $n \geq 30$ or plot shows no \textit{strong} skew or \textit{obvious} outliers.
\end{itemize}
\textbf{Do} \textit{(TInterval)}

\[\bar{x} \pm t* \frac{S_x}{\sqrt{n}},\;\;df = n - 1\]

\noindent\textbf{Conclude}
I am C\% confident that the interval from ... captures the true population mean of ...
\\~\\
\noindent\textbf{Using interval:} Since ... was/was not captured in the (C\%) confidence interval ... (for ...), it is/is not plausible that ... Therefore, the confidence interval provides/does not provide convincing evidence that ...
\\~\\
\noindent\textbf{Sample size needed:} $n \geq (\frac{z*\sigma}{ME})^2$
\\~\\
\noindent\hyperlink{toc}{Back!}
\newline\hrule

\end{document}